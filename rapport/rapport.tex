\documentclass[12pt, a4paper, onecolumn, twoside,french,cleardoublepage=plain,openany]{article}
\usepackage[french]{babel}
\usepackage[a4paper]{geometry}
\usepackage[utf8]{inputenc}
\usepackage[T1]{fontenc}
\usepackage{indentfirst} % Linéa sur les premiers paragraphes
\usepackage[babel=true]{csquotes} % csquotes va utiliser la langue définie dans babel
\usepackage{graphicx} % pour inclure les images
\usepackage[fleqn]{amsmath} % pour certains signes mathématiques
\usepackage{siunitx} % les unités SI
\usepackage{amssymb} % pour le signe de convolution
\usepackage{moreverb} % pour le code c, cpp...
\usepackage[pdfusetitle]{hyperref} % Pour les liens cliquables et metadata généré
\usepackage[french,boxed,lined,onelanguage]{algorithm2e}
\SetAlCapSkip{1em} % Marge entre l'algo et le caption
%\SetAlCapNameSty{textit} % style du texte du caption des algos
\usepackage[toc,page]{appendix} % Annexes
\usepackage{fancyhdr} % Pour \lhead, \rhead... 
\usepackage[font={it}]{caption}
\usepackage[backend=bibtex,style=alphabetic,sorting=debug]{biblatex}
\usepackage{multirow} % Pour colonnes multiples des tableaux
\usepackage{longtable} % Pour longs tableaux
\usepackage{array} % Pour \texttt sur tout une colonne
\usepackage{xcolor} % Pour éviter que footnote ne bug...
\usepackage{footnote} % Pour les footnotes dans les tableaux
\makesavenoteenv{tabular} % Pour les footnotes dans les tableaux
\usepackage{tabularx}
\usepackage{pdfpages} % Include des pdfs
\usepackage[nottoc,numbib]{tocbibind} % Pour faire apparaitre la biblio. dans le sommaire
\usepackage[]{minitoc} % Pour les sommaires intermédiaires
\usepackage{etoolbox} % Pour le toggle (pour afficher les annexe ou non)
\usepackage[acronym,toc,shortcuts]{glossaries}
\usepackage[french]{cleveref} 
\usepackage{booktabs}
\bibliography{bibiographie.bib}
\newcommand{\code}{\texttt}

\begin{document}

%\title{}
%\author{Maël Valais}
%\date{\today}
%\maketitle
%\tableofcontents

\section{Introduction}\label{introduction}
La logistique de gestion de crise humanitaire se distingue de la logistique classique par les contraintes humaines qu'elle impose. 
Lors d'une crise humanitaire, il arrive que les personnes ne soient plus atteignables
%...

\section{État de l'art}
\subsection{L'approche par couverture : le CTP} \label{CTP}
Permet de résoudre 
\subsection{Minimisation des coûts et facteurs humains : le CCVRP} \label{CCVRP}
Comme l'explique l'article \cite{campbell_routing_2008}, les approches classiques de calcul des tournées se basent sur le coût de transport, sans prendre en compte les aspects humains. Il est notamment expliqué que %...

Le CCVRP est issu de cette réflexion %...


\begin{table}[h] \centering
\begin{tabular}{@{}llll@{}}
\toprule % utilise \usepackage{booktabs}
 & CTP avec VRP & CTP avec CCVRP & Écart relatif \\ \midrule
Nombre de camions & 2 & 2 & +0,0\% \\
Distance parcourue & 17,09 & 17,09 & +0,0\% \\
Somme des temps d'arrivée & 15,07 & 15,07 & +0,0\% \\
Temps d'arrivée maximal & 7,19 & 7,19 & +0,0\% \\ \bottomrule
\end{tabular}
\caption{Avec deux camions disponibles}
\label{deux_camions}
\end{table}

\begin{table}[h] \centering
\begin{tabular}{@{}llll@{}}
\toprule % utilise \usepackage{booktabs}
 & CTP avec VRP & CTP avec CCVRP & Écart relatif \\ \midrule
Nombre de camions & 2 & 3 & +50,0\% \\
Distance parcourue & 17,09 & 22,17 & +29,7\% \\
Somme des temps d'arrivée & 15,07 & 12,12 & -19,6\% \\
Temps d'arrivée maximal & 7,19 & 4,24 & -41,0\% \\ \bottomrule
\end{tabular}
\caption{Avec trois camions disponibles}
\label{trois_camions}
\end{table}

\begin{table}[h] \centering
\begin{tabular}{@{}llll@{}}
\toprule % utilise \usepackage{booktabs}
/ & CTP avec VRP & CTP avec CCVRP & Écart relatif \\ \midrule
Nombre de camions & 2 & 4 & +100,0\% \\
Distance parcourue & 17,09 & 24,18 & +41,5\% \\
Somme des temps d'arrivée & 15,07 & 12,09 & -19,8\% \\
Temps d'arrivée maximal & 7,19 & 4,24 & -41,0\% \\ \bottomrule
\end{tabular}
\caption{Avec quatre camions disponibles}
\label{quatre_camions}
\end{table}

\section*{Annexes}
\subsection{Modèle mathématique du CTP-CCVRP}\label{modele}
\begin{equation}
Min \sum_{i=0}^{m}\sum_{j=0}^{m}\sum_{k=1}^{l}c_{ij}x_{ijk}
\end{equation}

s. t. :

\begin{equation}
\sum_{i=0}^{m}x_{ijk} = y_{jk}   j\in\left\{1,2,...,m\right\}, k\in\left\{1,2,...,l\right\}
\end{equation}

\begin{equation}
\sum_{i=0}^{m}x_{jik} = y_{jk}   j\in\left\{1,2,...,m\right\}, k\in\left\{1,2,...,l\right\}
\end{equation}

\begin{equation}
\sum_{j=0}^{m}x_{0jk} = 1   k\in\left\{1,2,...,l\right\}
\end{equation}

\begin{equation}
\sum_{j=0}^{m}x_{j0k} = 1   k\in\left\{1,2,...,l\right\}
\end{equation}

\begin{equation}
\sum_{j=1}^{m}\sum_{k=1}^{l}\alpha_{ij}D_{isjk} \geq d_{is}    i\in\left\{1,2,...,n\right\}, s\in\left\{1,2,...,t\right\}
\end{equation}

\begin{equation}
\sum_{i=1}^{n}\sum_{s=1}^{t}w{s}D_{isjk} \leq Q_{k}y_{jk}    k\in\left\{1,2,...,l\right\}, j\in\left\{1,2,...,m\right\}
\end{equation}

\begin{equation}
\sum_{s=1}^{t}\sum_{i=1}^{n}\sum_{j=1}^{m}w{s}D_{isjk} \leq Q_{k}    k\in\left\{1,2,...,l\right\}
\end{equation}

\begin{equation}
u_{ik} - u_{jk} + (m + 1)x_{ijk} \leq m   i,j\in\left\{1,2,...,m\right\}, k\in\left\{1,2,...,l\right\}
\end{equation}

\begin{equation}
x_{ijk} \in\left\{0,1\right\}   i,j\in\left\{1,2,...,m\right\}, k\in\left\{1,2,...,l\right\}
\end{equation}

\begin{equation}
y_{jk} \in\left\{0,1\right\}   j\in\left\{1,2,...,m\right\}, k\in\left\{1,2,...,l\right\}
\end{equation}

\begin{equation}
u_{ik} \geq 0   i\in\left\{1,2,...,m\right\}, k\in\left\{1,2,...,l\right\}
\end{equation}

\begin{equation}
D_{isjk} \geq 0   i\in\left\{1,2,...,n\right\}, s\in\left\{1,2,...,t\right\}, j\in\left\{1,2,...,m\right\}, k\in\left\{1,2,...,l\right\}
\end{equation}



\begin{equation}
u_{ik} - u_{jk} + Q_{k}x_{ijk} \leq Q_{k} - \sum_{s=1}^{t}\sum_{h=1}^{n}w_{s}D_{hsjk}   i,j\in\left\{1,2,...,m\right\}, k\in\left\{1,2,...,l\right\}
\end{equation}


\end{document}